\documentclass[10pt,oneside]{book}

\pagestyle{plain}
\usepackage{graphics}
\usepackage{makeidx}
\usepackage{nomencl}


% This is the same geometry as in the .dot file provided by Liverpool
\usepackage[top=2.54cm, bottom=1.27cm, left=3.05cm, right=3.05cm]{geometry}

% disables section and subsection numbering but leaves chapter numbering intact
\setcounter{secnumdepth}{0} 

% Arial is just a very cheap copy of Helvetica, so here is the real deal.
% UoL demands the use of a sans-serif font on blocks of text, so let's comply
\usepackage[T1]{fontenc}

% Helvetica doesn't really need to be set specifically, because pslatex
% includes setting the sans-serif font to Helvetica
\usepackage[scaled]{helvet}

% same as Times, but uses a specially narrowed Courier. This is preferred over 
% Times because of the way it handles Courier.
\usepackage{pslatex}
\renewcommand*\familydefault{\sfdefault}

% DANGER: Underlining section titles is considered bad typography. We
% do it anyway because UoL likes it

% titlesec seems to work better than sectsty for this horrible purpose 
\usepackage{titlesec}
\titleformat{\chapter} {\fontsize{14}{14}\rmfamily\bfseries\scshape\centering}{\chaptertitlename\ \thechapter.}{.5em}{}{} 
\titleformat{\section} {\fontsize{12}{12}\rmfamily\bfseries}{\thesection}{.5em}{\underline}{}
\titleformat{\subsection} {\fontsize{12}{12}\rmfamily\bfseries}{\thesection}{.5em}{\underline}{}
\titleformat{\subsubsection} {\fontsize{12}{12}\rmfamily\bfseries}{\thesection}{.5em}{\underline}{}


% --- Colors
% UoL does not explicitly call for these, but they at least make it possible to
% tell a link/URL apart from normal text
\usepackage{color}
\definecolor{DarkBlue}{rgb}{0,0.03,0.25}

\usepackage[colorlinks]{hyperref}
\hypersetup{colorlinks=true,linkcolor=DarkBlue,anchorcolor=DarkBlue,citecolor=DarkBlue,filecolor=DarkBlue,urlcolor=DarkBlue}

\usepackage[pdftex]{graphicx}


% --- Bibliography-style (UoL demands Harvard for MSc but doesn't care much about that for PhD)
\usepackage{natbib}
\bibliographystyle{agsm}

% Without this, the bibstyle would use \harvardurl, which breaks
% on special characters such as _ in URLs
\usepackage{url}
\renewcommand{\harvardurl}{URL: \url}

% Put a , between author name and year
\bibpunct{(}{)}{,}{a}{,}{;}

% Double spacing was in the .dot file on the normal paragraph text
\usepackage{setspace} 
\doublespacing

\makeindex
\makeglossary
\begin{document}


\title{Your dissertation's title}
\author{Your name}
\date{XYZ 2010}
\maketitle 
\frontmatter

\chapter{Abstract}

Put some abstract text here.

\tableofcontents

% Contents not put in table of contents by default so add it separately

\addcontentsline{toc}{chapter}{Contents}

\listoffigures


% List of figures  not put in table of contents by default so add it separately

\addcontentsline{toc}{chapter}{List of Figures}


\chapter{Acknowledgement}

The author wishes to thank the Flying Spaghetti Monster for putting him into this world via His Noodly Appendage.


\printglossary

\addcontentsline{toc}{chapter}{Nomenclature}


\mainmatter

\chapter{First Chapter}
\section{First Section}

Section text.

\subsection{Some subsection}

A subsection.



\appendix
\chapter{Appendix}
Appendices are usually labelled with letters separate to ordinary chapters.


% Switch back to single spacing so things don't look so ugly in these sections
\singlespacing

% Don't forget to supply a dissertation.bib file with your bibliography in
% BibTex format
\bibliography{dissertation}
\addcontentsline{toc}{chapter}{Bibliography}

\printindex
\addcontentsline{toc}{chapter}{Index}


\end{document}
